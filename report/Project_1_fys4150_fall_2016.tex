\documentclass[10pt]{article}
\usepackage{mathtools}
\usepackage{amsmath}

\begin{document}
\begin{abstract}
Andrei's part.
\end{abstract}

\part*{Introduction}
In this project we are dealing with the Poisson equation. This is a second order inhomogeneous differential equation of elliptic type. In this project we have a very simple case of equation in one dimension. We are going to discretize this equation and rewrite it in form of system of linear equations. We can then use some methods from linear algebra to find a solution. This kind of equations is common in physics. For example it can describe an electrostatic or gravitational fields.  \\* 
Below is a brief description of the project structure. \\* 
In this project we take a look at Poisson equation for electrostatic potential generated by a localized charge distribution. The mathematical details are discussed in Part 1. Problem formulation.\\* 
After we formulated the problem we consider some general and most useful methods we can use to solve it in the Part 2. Theory and methods. We are going to use linear algebra methods for the problem. A short description of the Gaussian elimination and LU decomposition will be given there, as well as some discussion of the pros and cons for each of them.\\* 
In Part 3. the results are presented. We took the most interesting (to our point of view) data and present it in tables and graphs. \\* 
The next part is discussion of the results. Here we evaluate our results as well as methods used to obtain them. \\* 
The last part of the report is Conclusion. Here we sum up all said above and come up with the possible topics for the further studies  in this topic. 

\part*{Part 1. Problem formulation}
The problem we deal with is a second order differential equation (DE) with inhomogenious term. Generally it can be written in a following form:
\[
\frac{d^2y}{dx^2}+k^2(x)y = f(x),
\]
where $y$ is unknown function of $x$, $f(x)$ is inhomogeneous term and $k^2$ is a real function.\\* 
Our DE is a classical equation of electrostatic potential, which is generated by some known charge distribution. 
\[
\nabla^2 \Phi = -4\pi \rho ({\bf r}).
\]
where $\Phi$ is electrostatic potential and $\rho ({\bf r})$ is charge distribution. \\* 
This is equation for three dimensions. We simplify this to one dimension equation in $r$ assuming the functions for electrostatic potential and charge  being spherically symmetric. After some simple variable substitution we get a general form for one-dimensional Poisson equation. We also need some boundary conditions for the equation. In this project we have a Dirichlet boundary conditions. All together it will then look as follows:
\begin{equation}\label{equ:one}
-u''(x) = f(x), \hspace{0.5cm} x\in(0,1), \hspace{0.3cm} u(0) = u(1) = 0.
\end{equation}

Here we need to introduce a disctretization of the derivative. In order to do this we first discretize a domain and define a so-called grid points for the $ x $. We have $x\rightarrow x_{i}$ and for the each of a grid points we have a discretize value of the function in this point $u(x_{i})\rightarrow v_i$. For simplisity we have the same distance between all grid point and this is so-called step lenght $ h $. Each grid poin can now be calculated as $x_i=ih$. The valuse $ h $ can be calculated as $h=1/(n+1)$. And the boundary condition is now defined as $v_0 = v_{n+1} = 0$. $ n $ is number of the steps in discretization. The discretized Poisson equation then will be
\[
   -\frac{v_{i+1}+v_{i-1}-2v_i}{h^2} = f_i  \hspace{0.5cm} \mathrm{for} \hspace{0.1cm} i=1,\dots, n,
\]

 $f_i$ here is the actual function at grid points $f(x_i)$. \\* 
 If we now run $ i $ from $ 1 $ to $ n $ it's easy to show that our discretized equation can be written as a set of linear algebra equations (SLAE):
 \\* 
  \\* 
\begin{equation}
\begin{cases}
0 - v_{1}-v_{2} = f_1h^2 \\ -v_{1}+2v_{2}-v_{3} = f_1h^2  \\ 
-v_{2}+2v_{3}-v_{4} = f_1h^2  \\ \dots  \\ -v_{n-1}+2v_{n}-0= f_1h^2 \\
\end{cases},
\end{equation}

here we use $v_0 = v_{n+1} = 0$ conditions and miultiply the right hand side of the equation with an $ h^2 $. To simplify this we now introduce a new function $\tilde{b}_i=h^2f_i$. This system can be now written in a matrix form as:
\begin{equation}\label{equ:three}
   {\bf A}{\bf v} = \tilde{{\bf b}},
\end{equation}
where ${\bf A}$ is an $n\times n$  tridiagonal matrix  
\begin{equation}
    {\bf A} = \left(\begin{array}{cccccc}
                           2& -1& 0 &\dots   & \dots &0 \\
                           -1 & 2 & -1 &0 &\dots &\dots \\
                           0&-1 &2 & -1 & 0 & \dots \\
                           & \dots   & \dots &\dots   &\dots & \dots \\
                           0&\dots   &  &-1 &2& -1 \\
                           0&\dots    &  & 0  &-1 & 2 \\
                      \end{array} \right),
\end{equation}
In our case we have a tridiagonal matrix with the same elements along the main diagonal (as well as to other diagonals). However we will look at more general case, the tridiagonal matrix with different elements along diagonals. 
\begin{equation}
    {\bf A} = \left(\begin{array}{cccccc}
                           b_1& c_1 & 0 &\dots   & \dots &\dots \\
                           a_1 & b_2 & c_2 &\dots &\dots &\dots \\
                           & a_2 & b_3 & c_3 & \dots & \dots \\
                           & \dots   & \dots &\dots   &\dots & \dots \\
                           &   &  &a_{n-2}  &b_{n-1}& c_{n-1} \\
                           &    &  &   &a_{n-1} & b_n \\
                      \end{array} \right)\left(\begin{array}{c}
                           v_1\\
                           v_2\\
                           \dots \\
                          \dots  \\
                          \dots \\
                           v_n\\
                      \end{array} \right)
  =\left(\begin{array}{c}
                           \tilde{b}_1\\
                           \tilde{b}_2\\
                           \dots \\
                           \dots \\
                          \dots \\
                           \tilde{b}_n\\
                      \end{array} \right).
\end{equation}
This kind of SLAE can be solved using different linear algebra methods. In the next section we will discuss some of them.
\part*{Part 2. Theory and methods}
\subsubsection*{2.1 Gaussian Ellinimation.} 
When we have a SLAE formulated as in (3), we would like to change the matrix $ A $ so it become all zeroes below the diagonal. The most straight forward way to do it is Gauss elimination. 
The algorythm for the Gaussian elimination can be described as to main operations the forward substitution and the backward substitution. In t	he first part, the forward substitution, we would transform the original matrix $ A $ to some other matrix, with zeroes below the main diagonal. This can be done using some 

\end{document}
